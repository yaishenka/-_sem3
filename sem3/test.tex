%%%%%%%%%%%%%%%%%%%%%%%%%%%%%%%%%%%%%%%%%%%%%%%%%%%%%%%%%%%%
\newcommand{\coursename}{Математический анализ 1 к. 2 с.}
\newcommand{\compiledby}{Даниил Гагарин$\frownie$в, gagarinovdaniil@gmail.com } % укажите свое имя
\newcommand{\coursedate}{Весна 2018 г.}
\input{../classes_style.tex}
%%%%%%%%%%%%%%%%%%%%%%%%%%%%%%%%%%%%%%%%%%%%%%%%%%%%%%%%%%%%

\begin{document}
\section{Семинар №3}

\subsection{Пространство $\R ^n$}
 
\Def Метрическое пространство $\R ^n$ ---  множество элементов (точек) $ x \eq \underbrace{ \left( x_1,\, x_2,\ ... ,\, x_n \right)}_{\textcolor{red}{x_i - \text{координаты}}}$ с заданным  \\ расстоянием между ними:

~~~$\rho(x,y) \eq \sqrt{\sum \limits_{i=1}^{n}  \; \left( x_i - y_i \right)^2} $\ \textcolor{red}{(евклидова метрика)}


\underline{Свойства $\rho(x,y)$ :}

~~\parbox[t]{0.95\linewidth} {1)$\rho(x,y) \Ge 0$ ; \ $\rho(x,y)\!=\!0 \Leftrightarrow x \eq y$;

2)$\rho(x,y)\eq\rho(y,x);$
 
3)$\forall z \in \R^n \holds \rho(x,y) \Le \rho(x,z) + \rho(z,y)$;\ \textcolor{red}{$\left(  \text{неравенство } \bigtriangleup \right) $ }}

\Note $\R$ можно рассматривать как метрическое пространство $\R^n$ при $n \eq 1$; в нем $\rho(x,y) \eq \left| y-x \right|$

\Def $\underset{\textcolor{red}{\R^n \text{ --- линейное пространство}}} {\text{В } \R^n \text{вводятся операции}} $ $x+y \eqdef \left(x_1+y_1, ...,x_n+y_n\right)$ и $\lambda x \eqdef \left( \lambda x_1, ..., \lambda x_n \right)$

\Def (Шаровая) $\varepsilon$ --- окрестность  точки $x^{(0)}$ : $\Ue (x^{(0)}) \eq \left\{x \ | \ \rho(x,x^{(0)})< \varepsilon \right\} $ 

\begin{figure}[h]
    \includegraphics[width=0.1\textwidth]{png1.png}
\end{figure}

\Note Аналогично можно ввести прямоугольную, квадратную и другие окрестности.

\Def $\Uoe(x^{(0)})\eq \Ue (x^{(0)}) \setminus \left\{ x^{(0)} \right\} $

\Def Последовательность $\left\{ x^{(m)} \right\} $ --- отбражение $\N  \rightarrow \R^n$ $\textcolor{red}{\left( \forall m \in \N \ \text{поставлена в соответсвие точка}\right)}$

\Def $\lim\limits_{m \to \infty} x^{(m)} \eq x^{(0)}$, если $\forall \varepsilon > 0 \ \exists N \in \N : \forall m \Ge N \holds \rho (x^{(m)}, x^{(0)})< \varepsilon$ \ \ \textcolor{red}{$\left( x^{(m)} \in \Uoe(x^{(0)}) \right)$}

\Note $\lim\limits_{m \to \infty} x^{(m)} \eq x^{(0)} \Leftrightarrow \lim\limits_{m \to \infty} \underbrace{ \rho (x^{(m)}, x^{(0)})}_{\textcolor{red}{\substack {\text{числовая} \\ \text{последовательность}}}}= 0 \Leftrightarrow \underbrace{\lim\limits_{m \to \infty} x_i^{(m)} = x_i^{(0)}  }_{\textcolor{red}{\substack {\text{пределы n числовых} \\ \text{последовательностей}}}} , \ i\eq 1, \ldots, n$

\Def $\lim\limits_{m \to \infty} x^{(m)} = \infty , \text{если } \rho (x^{(m)}, 0) \eq \infty $

\Problem{} Д-ть: $\lim\limits_{m \to \infty} \rho(x^{(m)}, 0) \eq \infty \Leftrightarrow \forall a \in \R^n \holds 
\lim\limits_{m \to \infty} \rho(x^{(m)}, a) \eq \infty$

\Proof  

~~ \parbox[t]{0.95\linewidth} {"$\Rightarrow$" пусть $\lim\limits_{m \to \infty} \rho(x^{(m)}, 0) \eq \infty \Rightarrow \forall a \in \R^n \holds \rho(x^{(m)}, a) + \rho(a, 0) \Ge \rho(x^{(m)}, 0) \Rightarrow \\ \rho(x^{(m)}, a) \Ge 
\rho(x^{(m)}, 0) - \rho(a,0) \Rightarrow \rho(x^{(m)}, a) \underset {m \to \infty} {\longrightarrow} \infty$;

"$\Leftarrow$" пусть $\forall a \in \R^n \holds \lim\limits_{m \to \infty} \rho(x^{(m)}, a) \eq \infty$. Возьмем $a \eq 0 \in \R^n \Rightarrow  \lim\limits_{m \to \infty} \rho(x^{(m)}, 0) \eq \infty$; \Endproof}

\Def $\left\{ x^{(m)} \right\}$ называется ограниченной, если $\underline{\text{числовая}}$ последовательность $\left\{ \rho(x^{(m)}, 0)  \right\}$ ограничена.

\Th{Больцано-Вейерштрасса}  Из $\forall$ ограниченной последовательность можно выделить сходящуюся подпоследовательность.

\Th{Критерий Коши сх. посл-ти} $\left\{ x^{(m)} \right\}$ сх. $\Leftrightarrow \forall \varepsilon \!>\! 0 \ \exists N \! \in \! \N \! :  \!$ $\forall m \Ge N, \forall p \in \N \holds \rho (x^{(m)}, x^{(m+p)}) < \varepsilon$

%%%%%%%%%%%%%%%%%%%%%%%%%%%%%%%%%%%%%%%%%%%%%%%%%%%%%%%%%%%%%%
\newpage
\subsection {Классификация множеств в $\R^n$}

\fbox{\textbf{\RNumb{1}}} \textbf{Открытые множества.}

\Def х называется внутренней точкой м-ва $E \subset\R^n$, если $\exists \Ue(x) \subset E$.

\Def Внутренность мн-ва $E \subset\R^n$ : $intE \eq \left\{ x | x \text{--- внутренняя т. } E \right\}$.

\Example\!: $int[1;2]=(1;2)$ \textcolor{red}{(внутренность отрезка --- интервал)}

\Def Мн-во $E \subset \R^n$ называется открытым, если $\forall x \in E$ является внутренней т. E, т.е. \textcolor{red}{\fbox{$intE \eq E$}}

\Note  $\varnothing, \R^n$ --- пустые множества. \textcolor{red}{\fbox{!}}

\todo

\Th{} A, B --- открытые мн-ва $\Rightarrow \underbrace{A \cup B}_{\text{а}}$ открытое и $\underbrace{A \cap B}_{\text{б}}$ открытое.

\Proof

~~ \parbox[t]{0.95\linewidth}{а) пусть $x \in A \cup B \Rightarrow x \in a \text{ или } x \in B$. Если $x \in A$, то \ $\exists \Ue(x) \subset A \subset A \cup B \Rightarrow x \text {--- внутренняя точка } A \cup B$. \\  Аналогично при $x \in B$.

б) если $A \cap B \eq \varnothing$ очев. Пусть $A \cap B \neq \varnothing$. Пусть $x \in A \cap B \rightarrow x \in A$ и $x\in B$.  \\ x --- внутр. для A $\Rightarrow \exists \U_{\varepsilon_A}(x) \subset A$; x --- внутр. для B $\Rightarrow \exists \U_{\varepsilon_B}(x) \subset B$. Возьмем $\varepsilon \eq min \left\{ \varepsilon_A, \varepsilon_B \right\}$.  
\\ Тогда $\Ue(x) \subset A \cap B \Rightarrow x$ --- внутренняя для $A \cap B$. \Endproof
}

\todo

\Consequence{} \parbox[t]{0.95\linewidth}{$\cup \ \underline{\text{любого}}$ числа открытых множеств --- открытое множество.

$\cap \ \underline{\text{конечного}}$ числа открытых множеств --- открытое множество$^{ \textbf{*}}$}

\Note Для бесконечного числа открытых множеств \textbf{*)} выполняется не всегда.

\Def $\forall$ открытое мн-во, содержащее т.x, называется окрестностью т.x и обозн. $\U(x)$. Аналогично $\Uo(x)$. \textcolor{red}{\fbox{!}}

\smallskip

\fbox{\textbf{\RNumb{2}}} \textbf{Замкнутые множества.}

\Def $x \in \R^n$ --- т. прикосновения мн-ва $E \subset \R^n$, если $\forall \U(x) \cap E \neq \varnothing$.

\Def Замыкание множества $E \subset \R^n$ : $\overline{E} \eq \left\{ x | x \text{ --- т. прикосновения E}\right\}$

\todo

\Def Мн-во $E \subset \R^n$ называется замкнутым, если  $\overline{E} \eq E$

\Note $\varnothing$, $\R^n$ --- замкнутые множества \textcolor{red}{(и одновременно открытые)        
 \fbox{!}}

\Def $\R^n \backslash E$ называется дополнением множества E.

\Th{} E открыто $\Leftrightarrow \R^n \backslash E$ замкнуто. \textcolor{black}{$\left( \text{E замкнуто $\Leftrightarrow \R^n \backslash E$ открыто} \right)$}

\Th{законы де Моргана} 

~~\parbox[t]{0.95\linewidth}{$\R^n \backslash \big( \bigcup\limits_i E_i \big) \eq \bigcap\limits_i \big( R^n \backslash E_i \big)$

$\R^n \backslash \big( \bigcap\limits_i E_i \big) \eq \bigcup\limits_i \big( R^n \backslash E_i \big)$}

\Consequence{} A,B --- замкнутые множества $\Rightarrow A \cap B$ замкнуто и $A \cup B$ замкнуто.

\Consequence{} \parbox[t]{0.95\linewidth}{$\cap \ \underline{\text{любого}}$ числа замкнутых множеств --- замкнутое множество.

$\cup \ \underline{\text{конечного}}$ числа замкнутых множеств --- замкнутое множество$^{ \textbf{*}}$}

\Note Для бесконечного числа замкнутых множеств \textbf{*} выполняется не всегда.

\Example  $\bigcup\limits_{n=1}^{\infty} \underbrace{\left[ -1 + \frac{1}{n}, 1 - \frac{1}{n} \right]}_{\text{зам.}} \eq \underbrace{(-1, 1)}_{\text{не зам.}}$

%%%%%%%%%%%%%%%%%%%%%%%%%%%%%%%%%%%%%%%%%%%%%%%%%%%%%%%%%%%%%%
\newpage

\fbox{\textbf{\RNumb{3}}} \textbf{Изолированные и предельные точки.}

\Def $x \in E$  --- изолированная точка множества $E \subset \R^n$, если $\exists \Uoe(x) : \Uoe(x) \cap E \eq \varnothing$.

\todo

\Def x --- предельная точка множества $E \subset \R^n$, если $\forall \Uo(x) \cap E \neq \varnothing$

\Note x --- предельная точка E $\Leftrightarrow \exists$ последовательность Гейне ${x^{(m)}} \subset E$, сходящаяся к x.

\Note \parbox[t]{0.95\linewidth}{$\forall$ мн-ва $E \subset \R^n \holds$ \{т-ки прикосновения\} = 
\{предельные т-ки\}  $\cup$ \{изолированные т-ки\} 

причем \{предельные т-ки\}  $\cap$ \{изолированные т-ки\} $\eq \varnothing$. }

\smallskip

\fbox{\textbf{\RNumb{4}}} \textbf{Границы множеств.}

\Def x --- граничная точка множества $E \subset \R^n$, если $\forall \U(x)$ содержит как точки из E, так и из $\R^n \backslash E$.

\Def Граница множества  $E \subset \R^n$ : $\partial E = \left\{ x  \text{ | x --- граничная точка } E \right\}$.

\todo  

\Examples E-? : \parbox[t]{0.95\textwidth}{
1)$\partial E \subset$ и $\partial E \neq E$ --- замкнутый круг

2)$\partial E \eq E$ --- окружность, точка

3)$\partial E \cap E \eq \varnothing$  --- открытый круг

4)$\partial E \eq \varnothing$  --- $\R^n$

5)$\partial E \supset E$ и $\partial E \neq E$ --- $\Q$,  \tikz{\draw (-1,-1) -- (-0.5,-0.5); \path[draw=red] (-0.75,-0.75) circle (1mm);} }

\smallskip

\fbox{\textbf{\RNumb{5}}} \textbf{Связность множества. Области и компакты.}

\Remind Непрерывная (параметрически заданная) кривая $	\Gamma \subset \R^n$ -- множество точек, заданное как непрерывное отображение отрезка $\alpha \Le t \Le \beta$: 

\parbox[t]{0.95\linewidth}{$ \Gamma = \left\{ x(t) = (x_1(t), \ldots, x_n(t)) | t \in [\alpha,\beta] \right\}$}

\Def Множество $E \subset \R^n$ называется \underline{линейно связным}, если $\forall$ его 2 точки можно соединить непрерывной кривой $\Gamma \subset E$.

\Note Множество, состоящее из 1 точки, считается линейно связным.

\todo

\Def Множество $E \subset \R^n$ называется областью, если оно открыто и линейно связно.

\Def Множество $E \subset \R^n : E \neq \varnothing $, называется компактом, если оно замкнуто и ограничено.

\Examples интервал -- область, отрезок --- компакт.

\smallskip

\subsection{Дополнение к множествам в $\R^n$}

\fbox{\textbf{\RNumb{1}}} \ \parbox[t]{0.95\linewidth}{\Def Множество $E \subset \R^n$ называется \underline{связным}, если  $\forall$ его разбиения $E=\underset{\substack {\cancel{\textbardbl} \\ \varnothing}}{A} \cup \underset{\substack {\cancel{\textbardbl} \\ \varnothing}}{B}$, $A \cap B = \varnothing \holds A \cap \overline{B} \neq \varnothing$.

\Example \parbox[t]{0.95\linewidth}{связного, но не линейно связного множества : $E \eq \Big\{ (x,y) | x \Ge 0, y = \left\{ \substack{\sin{\frac{1}{x}}, x\ne0 \\ 0, x \eq 0} \right\}$.

E --- график функции, имеющей точки разыва \RNumb{2} рода $\Rightarrow$ не линейно связное;

но $\{(0,0)\} \cap \overline{E \backslash \{(0,0)\}} \eq \{(0,0)\} \ne \varnothing \Rightarrow$ E связно.}

\Note E линейно связно $\Rightarrow$ E связно.}

%%%%%%%%%%%%%%%%%%%%%%%%%%%%%%%%%%%%%%%%%%%%%%%%%%%%%%%%%%%%%%

\newpage

\fbox{\textbf{\RNumb{2}}} \ \parbox[t]{0.95\linewidth}{ \Th{лемма Гейне-Бореля}  Пусть $K \subset \R^n$ --- компакт и $\left\{ G_k\right\}_{k=1}^\infty$ --- его $\overbrace {\text{покрытие}}^{K \subset \bigcup\limits_{k=1}^{\infty} G_k} $ \\ открытыми мн-ми. Тогда из $\left\{ G_k\right\}_{k=1}^\infty$  можно выделить \underline{конечное} подпокрытие $\left\{ G_k\right\}_{k=1}^m$ компакта K.

\Proof для $\R$  и случая $K \eq \left[a,b\right]$

Предположим обратное $\Rightarrow$ одну из половин $\left[a,\frac{a+b}{2} \right], 
\left[\frac{a+b}{2}, a \right]$ нельзя покрыть \\ конечным подпокрытием. Делим этот отрезок пополам и т.д. Получаем СВО с длиной $\rightarrow$ 0, \\ стягивающуюся к $x_0 \in [a,b]. \ x_0 \in$ одному из $G_k \Rightarrow$ начиная с некоторого шага деления отрезок целиком $\subset G_k$ ?! \Endproof

\underline{Примеры использования:} \parbox[t]{0.95\linewidth}{
	
1) доказательство Th Кантора о РН $f(x)$ на [a,b]

2)задача: доказать, что $\bigcup\limits_{k=1}^{\infty} \U_{\frac{1}{2^k}} (q_k) \neq \R$ }
} 

\Problem{\&2 9.2} Является ли область определения $f(x,y) = \sqrt{x \sin{y}}$ замкнутым? открытым? областью?

\todo

~~\parbox[t]{0.95\linewidth}{

\Statement $\overline{E} \eq intE \cap \partial E$

~~\parbox[t]{0.95\linewidth} {\Proof  Пусть $x \in \overline{E} \Rightarrow \forall \U(x)$ содержит $x_1 \in E$. Либо $\exists \U^{(1)}(x)$, целиком $\subset E$, либо

 $\forall \U(x)$ содержит $x_2 \notin  E$. В первом случае $x \in intE$, во втором x $\in \partial E$ \Endproof
 }

\Consequence{} $\partial E \subset E \Rightarrow \overline{E} \eq E. \left(  \overline{E} = \underbrace{intE}_{\substack {\cap \\ E}} \cup  \ \partial E  \right)$

Обозначим область определения $D$. $\partial D = \left\{ (0,y) \ | \ y \in \R \right\} \cup \left\{ (x,2\pi n) \ | \ x \in \R, n \in \Z \right\} \subset D \Rightarrow \overline{D} \eq D \Rightarrow $

$\Rightarrow  D$ замкнуто.

$\left\{ \substack {\text{ \normalsize $\partial D \subset D \Rightarrow D$ не является открытым $\Rightarrow D$ не область.}   \\ \text{ \normalsize $\partial D \ne \varnothing$ }\hfill } \right. \ $ \Note \text{D линейно связно} 
}

\Problem {\&1 13}  $f \in \mathbb{C} \; (\R), \ y_0 \in \R. \ \Prooff \ S = \left\{ x \; | \; f(x) > y_0 \right\}$ открыто.

~~\parbox[t]{0.95\linewidth} {
\textbf{План:} берем $x_0 \in S$, берем $\varepsilon = \frac {f(x_0)-y_0}{2}$, показываем, что $x_0 \in int S$.


}

\smallskip

\subsection{Функции многих переменных}

\Def $y = f(x) = f(x_1, \ldots, x_n)$ --- отображение $X \rightarrow \R$, где $X \subset \R^n$ --- множество  определения. 

Рассмотрим $f(x)$ на множестве $E \subset X$; пусть $x^{(0)}$ --- предельная точка $E$.

\parbox[t]{0.95\linewidth}{
	\Def (предел $f(x)$ в точке  $x^{(0)}$ по множеству $E$)
	
	$\lim\limits_{\substack {x \to x^{(0)} \\ x \; \in \; E}} f(x) \eq a$, если 
	$\left[ \substack { 
		\text{\normalsize по Гейне: $\forall \ \{ x^{(m)} \} \subset E \ \backslash \ $  
		 $\{ x^{(0)} \}: \  \lim\limits_{m \to \infty} x^{(m)} = x^{(0)} \holds$
		$ \lim\limits_{m \to \infty} f( x^{(m)}) \eq a $}
		\\
		\text{\normalsize по Коши: $\forall \varepsilon > 0 \ \exists \delta > 0 : \forall x \in 					\Uo_{\delta} (x^{(0)}) \cap E \holds | f(x) - a | < \varepsilon $}\hfill
	 } \right. $
}

\textbf{Обозначения:} $\lim\limits_{\substack {x \to x^{(0)} \\ x \; \in \; E }} f(x) \eq $
$\lim\limits_{\substack {\rho(x,x^{(0)}) \to 0 \\ x \; \in \; E}} f(x) \eq$
$\lim\limits_{\substack {x_1 \to x^{(0)}_1 \\ \vdots \\ x_n \to x^{(0)}_n \\ (x_1, \ldots, x_n) \in E} } f(x) $.

\Example функция Дирихле $D(x) \eq$ 
$\left\{ \substack { 
		\text{\normalsize $1, x \in \Q$}
		\\
		\text{\normalsize $0, x \in \mathbb{I}  $ \hfill}
 }\right. $ 
$\lim\limits_{\substack{ x \to 0 \\ x \in \Q}}D(x) \eq 1$, $\lim\limits_{\substack{ x \to 0 \\ x \in \mathbb{I}}}D(x) \eq 0$.

\parbox[t]{0.95\linewidth}{\Note в $\R^n$ сохраняются основные свойства пределов (\textbf{Th} об арифметический действиях, \\ свойства, связанные с неравенствами, понятия бесконечно малых и бесконечно больших функций).}

\Def (предел $f(x)$ в точке $x^{(0)}$)

~~Пусть $f(x)$ определена в некоторой $\Uo(x^{(0)}). \lim\limits_{x \to x^{(0)}} f(x) \ \ \ \ \eqdef 
\lim\limits_{\substack {x \to x^{(0)} \\ x \; \in \; \Uo(x^{(0)}) }} f(x)$.

%%%%%%%%%%%%%%%%%%%%%%%%%%%%%%%%%%%%%%%%%%%%%%%%%%%%%%%%%%%%%%

\newpage

\Th{} \parbox[t]{0.95\linewidth}{ Пусть $f(x)$ определена в некоторой $\Uo(x^{(0)}).  \lim\limits_{x \to x^{(0)}} f(x) \eq a \Leftrightarrow \forall E \subset X : $ \\
$x^{(0)}$ --- предельная точка $E \holds \lim\limits_{\substack {x \to x^{(0)} \\ x \; \in \; E }} f(x) \eq a$ }

$x^{(0)}$   $\Ue(x^{(0)})$






































 











\end{document}